\section{Syncers}
\label{sec:syncers}

anysyncd comes with two syncers (action modules) for different purposes:

\begin{description}
    \item[Anysyncd::Action::Rsync2] rsync-based synchronisation for
    uni-directional syncing of directories.  
    \item[Anysyncd::Action::Csync2] csync2-based
    \footnote{http://oss.linbit.com/csync2/} synchronisation for bi-directional
    syncing of directories with automatic conflict resolution
\end{description}

Note that some syncers may require additional work to be done outside of
anysyncd, e.g. exchange of ssh keys for an rsync via ssh. These steps are out
of scope for this document.

\subsection{General parameters}
\label{ssec:syncer-general}
The following configuration keys are supported in each sync configuration,
regardless of the chosen syncers.

\begin{description}
    \item[watcher] Specifies which directory is watched for inotify events.

    \item[cron] Allows to schedule regular syncs, additionally to the inotify-
        based ones. The given configuration string must be a crontab-style
        schedule-declaration (the first five fields of a crontab line)
    \item[waiting_time] Allows to override the time span in seconds, in which
        inotify events are aggregated after an inotify event occured. This
        defaults to 5 seconds, which is usually a good compromise between
        sync lag and efficient use of bandwith.
\end{description}

\subsection{Anysyncd::Action::Rsync2}

The rsync-syncer is aimed at the classical use-case, where one wants to sync
all files in a directory to another directory or host, every time a file in the
directory changes. It may call rsync2 multiple times, to make sure source
and destination are really in sync.

The syncer supports the following parameters additionally to those
stated in \ref{ssec:syncer-general}.

\begin{description}
    \item[from] Specifies the source, where to sync from. Should usually be
        identical to the directory specified as watcher, as described in
        \ref{ssec:syncer-general}

    \item[to] Specifies the destination, where to sync to. This should be a
        dest specification as used on a rsync commandline, e.g.
        user@host:/foo/bar/ for ssh.
\end{description}

\subsection{Anysyncd::Action::Csync2}

If changes could occur on both sides (even if only by accident) this syncer
can be used. It uses csync2 to synchronize source and destination, which is
able to cope with bi-directional changes, handle file deletions and detect
conflicts. 

The syncer is somewhat special insofar that it does a multi-step-syncing
which involves a local rsync and an activating on the remote system, in order to
make sure that the remote directory is replaced in one atomic operation. That
is because the syncer was designed for a use-case where it's required that the
directory is always in a consistent state. But this also means that it comes
with some requirements:

\begin{itemize}
    \item Some directories MUST be defined. See the parameters below for more
        information.
    \item The user running anysyncd must be able to ssh into the configured
        remote_hosts password-less.
    \item csync2 needs to be configured according to the csync2
        documentation.\footnote{http://oss.linbit.com/csync2/paper.pdf} 
\end{itemize}

The syncer supports the following parameters additionally to those
stated in \ref{ssec:syncer-general}.

\begin{description}
    \item[prod_dir] Specifies the source, where to sync from. Should usually be
        identical to the directory specified as watcher, as described in
        \ref{ssec:syncer-general}. Required.

    \item[csync_dir] Specifies the directory which is configured as include
        directory (target and source) in csync2. Required.

    \item[remote_hosts] Specifies the remote_hosts to which the source is
        synced. Needed for the remote activation ("commit") to work.
\end{description}

%% The exact process of a sync with this sycer is illustrated below:
% TODO: Draw and add drawing here

There is one limitation to consider as of writing this documentation: Since
anysyncd calls 'csync2 -x' unconditional, every time \textit{one} of the synced
directories changes, all csync2 synchronisation groups
are being processed.

