\section{Kennzeichnung von Priorität oder Kritikalität}
\label{sec:priorities}
Zur Kennzeichnung von Prioritäten, mit denen der Kunde verschiedene Themen
angehen sollte bzw.\ der Kritikalität von bei einer Analyse aufgedeckten
Problemen kann man die Titel von Abschnitten und Unterabschnitten farblich
hinterlegen, wie in den folgenden Unterabschnitten demonstriert.

Die dazu benutzten Farben sind zentral definiert und lassen sich auch
anderweitig nutzen, z.\,B.\ in einer einführenden Erläuterung der Farben:
Es stehen vier Varianten zur Verfügung, für
\colorbox{priorityfine}{\strut{}keinen Handlungsbedarf}] -- alles in Ordnung
-- sowie \colorbox{prioritylow}{\strut{}niedrige}],
\colorbox{prioritymed}{\strut{}mittlere}] und
\colorbox{priorityhigh}{\strut{}hohe}] Priorität.

\subsectionhigh{Abschnitt hoher Priorität oder Kritikalität}
Dieser Abschnitt beschreibt ein gravierendes Problem oder eine kurzfristig
durchzuführende Maßnahme.

\subsectionmed{Abschnitt mittlerer Priorität oder Kritikalität}
Dieser Abschnitt beschreibt ein mittelschweres Problem oder eine
mittelfristig durchzuführende Maßnahme.

\subsectionlow{Abschnitt niedriger Priorität oder Kritikalität}
Dieser Abschnitt beschreibt ein geringes Problem oder eine langfristig
durchzuführende Maßnahme.

\subsectionfine{Abschnitt ohne Handlungsbedarf}
In diesem Abschnitt ist alles in Ordnung, es wurden keine Probleme
festgestellt und/oder der Kunrde muß nichts tun.