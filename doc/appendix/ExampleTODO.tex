\section{Bemerkungen und FIXMEs}
\label{sec:todo}
Das \textsf{todonotes} Package erlaubt es in Verbindung mit den \verb|\NOTE|
und \verb|\FIXME| Makros auf einfache Weise Bemerkungen\NOTE{Beispiel Note}
mit Informationen sowie noch vor Vollendung des Dokuments auf jeden Fall zu
behebende Probleme\FIXME{Beispiel FIXME} einzufügen.
Benutzt man diese Befehle innerhalb von überschriften, so tauchen sie auch
im Inhaltsverzeichnis aus, wodurch man einen guten überblick erhalten kann,
z.\,B.\ bei einer Aufgabenverteilung auf mehrere Autoren.
Wie man dies nutzt, bleibt jedem Projekt selbst überlassen, die folgenden
Abschnitte zeigen ein Beispiel.

\subsection{Noch zu schreibender Abschnitt\FIXME{eho}}
Hier wird \verb|\FIXME| genutzt, um den Autor, der sich um den Abschnitt
kümmert zu kennzeichnen.

\subsection{Noch zu prüfender Abschnitt\CHECK{eho}}
Hier wird \verb|\CHECK| genutzt, um zu kennzeichnen, daß dieser Abschnitt
fertig geschrieben, aber noch nicht Korrektur gelesen ist -- dabei wird der
Autor weiterhin genannt, um auch ohne \texttt{git blame} beim Korrekturlesen
auf der Couch den Überblick zu behalten. 

\subsection{Fertiger Abschnitt\DONE{eho}}
Hier wird \verb|\DONE| genutzt, um zu kennzeichnen, daß dieser Abschnitt
fertig ist.
Hier kann sich ein Korrekturleser natürlich auch eintragen.

\subsection{Platzhalter für fehlende Grafiken}
Weiterhin kann man mit dem \verb|\missingfigure| Befehl Platzhalter -- und
Gedächtnisstützen -- für noch fehlende Graphiken einfügen, wie die in
Abbildung~\ref{fig:missingfigure-example}.

\begin{figure}[ht]
  \missingfigure{Erläuternde Graphik}
  \caption{Beispiel für Graphik-Platzhalter}
  \label{fig:missingfigure-example}
\end{figure}

Mit dem \verb|\listoftodos| Befehl kann man zudem einen Index aller
Bemerkungen und fehlenden Graphiken erzeugen lassen.
